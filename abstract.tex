We present results from deep \chandra\ observations of the large-scale filament extending SE from the massive Frontier Fields cluster MACS~J0717.5+3745. The filament was previously found to have a length of $\sim 19$ Mpc. Within the filament, about 2~Mpc away from the cluster center, there is a galaxy group with a mass of $\sim 5\times 10^{13}$~M$_\odot$ and a temperature of $\sim 3$~keV. This group is likely infalling for the first time towards the cluster. The filament is brightest in X-ray in the region between the group and the cluster. Here, the gas is found to have a temperature of $1.58_{-0.25}^{+0.51}$~keV and a density of $\sim 10^{-4}$~cm$^{-3}$. The filament density corresponds to a relatively high over-density of $\sim 100$ relative to the critical density of the Universe. If the X-ray emission is from the Warm-Hot Intergalactic Medium (WHIM), then the relatively high overdensity can be explained by the fact that we are probing only the densest part of the filament; the filament properties are consistent with numerical simulations and with the few other observational results reported to date. Alternatively, if the observed part of the filament is within the cluster's virial radius, the X-ray emission could be due to gas stripped from structures that collided with MACS~J0717.5+3745. This would require a filament inclination angle $\lesssim 60^\circ$ relative to the plane of the sky.
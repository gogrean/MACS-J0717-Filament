\section{Summary}
\label{sec:Summary}

MACS~J0717.5+3745 ($z=0.546$) is a massive galaxy cluster selected as one of the six Frontier Fields targets. The cluster is the most morphologically complex merger, being the site of collisions between at least four subclusters. SE of the cluster, there is a large-scale cosmic filament that was first reported by \citet{Ebeling2004}. Some of the subclusters involved in the merger have likely traveled along this filament before colliding with the previously existing structure. \citet{Jauzac2012} determined from optical data that the filament is $\sim 19$~Mpc long and $\sim 1.6$~Mpc wide. Part of the filament that is near the cluster is also visible at X-ray wavelengths. So far, the thermodynamic properties of large-scale filaments have only been studied in a handful of merging clusters \citep{Werner2008, Eckert2015, Bulbul2016}. Here, we used deep \chandra\ observations of MACS~J0717.5+3745 to study the properties of the large-scale filament extending SE of the cluster center, and those of the substructures along the filament. Below is a summary of our results:

\begin{itemize}
	\item The filament has a temperature of $1.58_{-0.25}^{+0.51}$~keV and a density of $\sim 10^{-4}$ cm$^{-3}$. These are consistent at the $90\%$ confidence level with the properties of the other filaments studied in X-ray \citep{Werner2008, Eckert2015, Bulbul2016}.
	\item The filament is over-dense by a factor of $\sim 250$ compared to the mean baryon density of the Universe.
	\item The total mass of the filament, assuming a constant density along its full length, is extremely large, $\sim 7\times 10^{14}$~M$_\odot$. For comparison, \citet{Eckert2015} estimated filament masses $<10^{14}$~M$_\odot$. However, our mass estimate,  which relies on the measured gas density, should be considered only an upper limit since the gas density is expected to decrease with increasing distance from the cluster.
	\item A little over $2$~Mpc SE of the cluster center, embedded within the filament, there is a galaxy group with a temperature of $\sim 3$~keV and an X-ray luminosity of $\sim 10^{43}$~erg~s$^{-1}$ in the energy band $0.1-2.4$~keV. The mass of the group is estimated to be $\sim 5\times 10^{13}$~M$_\odot$. This group is very likely to be approaching the cluster for the first time.
	\item Approximately $670$~kpc NW of the cluster center, there is a ram pressure-stripped core with a temperature of $\sim 7$~keV. This core is a remnant of a substructure that already collided with the cluster and traversed the densest regions of the ICM. On the N-NE edge of the core, we detect a density discontinuity that is likely associated with a cold front. However, projection effects and poor count statistics in this region prevent us from detecting a temperature jump that would be needed to confirm the nature of the density discontinuity.
\end{itemize}
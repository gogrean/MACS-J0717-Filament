GAO acknowledges support by NASA through a Hubble Fellowship grant HST-HF2-51345.001-A awarded by the Space Telescope Science Institute, which is operated by the Association of Universities for Research in Astronomy, Incorporated, under NASA contract NAS5-26555. RJvW is supported by a Clay Fellowship awarded by the Harvard-Smithsonian Center for Astrophysics.

This research made use of \textsc{APLpy}, an open-source plotting package for \textsc{Python} hosted at \url{http://aplpy.github.com}, and of \textsc{Astropy}, a community-developed core \textsc{Python} package
  for Astronomy \citep{astropy}. This research has also made use of NASA's Astrophysics Data System, and of the cosmology calculator developed by N. Wright \citep{Wright2006}. The \chandra\ data was obtained from the Chandra Data Archive, and analyzed using software provided by the Chandra X-ray Center (CXC) in the application packages \textsc{CIAO} and \textsc{ChIPS}. The surface brightness modeling used \href{https://github.com/gogrean/PyXel}{\textsc{PyXel}}, a \textsc{Python} open-source modeling package for X-ray astronomy. The surface brightness profiles were plotted using \textsc{matplotlib}, a \textsc{Python} library for publication quality graphics \citep{Hunter2007}.
  
 The optical data shown in the paper is based on observations made with the NASA/ESA Hubble Space Telescope, and obtained from the \href{http://hla.stsci.edu/}{Hubble Legacy Archive}, which is a collaboration between the Space Telescope Science Institute (STScI/NASA), the Space Telescope European Coordinating Facility (ST-ECF/ESA) and the Canadian Astronomy Data Centre (CADC/NRC/CSA).